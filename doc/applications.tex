\section{Applications}

We have successfully used \texttt{z3.rkt} in a number of applications, from
constraint-solving puzzles as illustrated in Section~\ref{sec:interactive}, to
bounded model checking and counterexample generation for functional programs.
Along the way, we have been pleasantly surprised by how well the power of Z3
and the expressiveness of Racket combine.

\subsection{Quantified Formulas Are Hard}

Z3 and other SMT solvers support lists and other recursive types. How would a
recursive function that operates over the entire list (say, to calculate its
length) be defined? Z3's macros are substitutions and do not support
recursion, so we are forced to look for alternatives. Our first intuition
might be to use a universal quantifier, as in Section~\ref{sec:derived}. Here
is a first cut at defining a function that calculates the length of an integer
list:

\begin{verbatim}
(declare-fun len ((List Int)) Int)
(assert (forall ((xs (List Int)))
  (ite (= xs nil)
       (= (len xs) 0)
       (= (len xs) (+ 1 (len (tail xs)))))))
\end{verbatim}

The problem with this is that solving assertions that use this requires model-
based quantifier instantiation (MBQI)~\cite{mbqi}, which while powerful can
also be quite slow. In our experience it is very easy to write a formula that
Z3's MBQI engine fails to solve in reasonable time. In general, SMT solvers
have a hard time dealing with quantified formulas containing linear
arithmetic, because for logics containg them there is no sound and complete
decision procedure~\cite{halpern91}.

So avoiding quantified formulas altogether seems like a good idea, but how do
we do that? The easiest way is to unroll and bound the recursion to a desired
depth~\cite{sat-recursive}. One way to do this is to define macros
\texttt{len-0}, \texttt{len-1}, \texttt{len-2}, \ldots \texttt{len-N}, where
each \texttt{len-i} returns the length of the list if it is less than or equal
to \texttt{i}, and \texttt{i} if it is greater than it.

\begin{verbatim}
(define-fun len-0 ((xs (List Int)))
  0)

(define-fun len-1 ((xs (List Int)))
  (ite (= xs nil)
       0
       (+ 1 (len-0 (tail xs)))))

(define-fun len-2 ((xs (List Int)))
  (ite (= xs nil)
       0
       (+ 1 (len-1 (tail xs)))))
...
\end{verbatim}

Our system makes defining a series of macros like this a cinch.

\begin{verbatim}
(define (make-length n)
  (smt:define-fun len ((xs (List Int))) Int
    (if (zero? n)
        0
        (ite/s (=/s xs nil/s)
               0
               (let ([sublen (make-length (sub1 n))])
                 (+/s 1 (sublen (tail/s xs)))))))
  len)
\end{verbatim}

\texttt{(make-length 5)} returns an SMT function that works for lists of up to
length 5, and returns 5 for anything bigger than that. Note how freely the
Racket \texttt{if} and \texttt{let} forms are mixed into the SMT body. These
constructs are evaluated at definition time, meaning that this definition
reduces to the series of macros defined above, except much more
concisely\footnote{In some applications it might be useful to start with
functions that handle small lists and then work our way up as Z3 keeps proving
facts for bigger and bigger list lengths. Such cases can either be implemented
by rewriting formulas as appropriate~\cite{sat-recursive}, or by incrementally
adding macros to handle larger lengths.}

Along the same lines, it is possible to define bounded recursive functions to
reverse, append and filter lists based on a predicate.
