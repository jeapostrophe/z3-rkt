\section{Conclusions and Future Work}

In this paper, we have presented \texttt{z3.rkt}, which lets users interact with
an SMT solver programmatically. We have demonstrated through an example the
simplicity and usefulness of such an interaction. The power of \texttt{z3.rkt}
comes from the facilities provided by Racket to build abstractions on top of the
SMT-solving capabilities of Z3. From the user's perspective, the integration is
seamless and fully transparent.

Our implementation is open source and freely available at
\begin{center}
\url{http://www.cse.iitk.ac.in/users/karkare/code/z3/}
\end{center}

\texttt{z3.rkt} is still a work in progress, and in the near term we plan to
achieve the following:

\begin{itemize}
\item Support for more of Z3 constructs, including external theories and procedure abstractions
\item New abstractions, guided by practical use cases
\item Work on model checking and debugging functional programs
\item Possible integration with other SMT solvers
\end{itemize}

The current interface requires expression-based manipulations like
\texttt{unquote} for programmatic use. We wish to experiment with moving Racket
and Z3 bindings into the same namespace, cleaning up the syntax further and
allowing higher-order functions like \texttt{map} and \texttt{apply} to be used
instead.

In the long term, we hope the community will find this system useful and
will contribute to the project to solve large practical problems.
