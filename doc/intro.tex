\begin{abstract}
 We present a work-in-progress system that integrates the Z3 SMT solver with the
 Racket programming language. The system defines a programmer's interface in
 Racket that makes it easy to harness the power of Z3 to discover solutions to
 logical constraints. The interface format, although in Racket, retains the
 structure and brevity of the SMT-LIB format. This system is expected to be
 useful for a wide variety of applications, from simple constraint solving to
 writing tools for debugging, verification, and automatic test generation for
 functional programs.
\end{abstract}

\section{Introduction}
\label{sec:motiv}

Satisfiability Modulo Theories (SMT) solvers let programmers specify constraints
over booleans, integers, pure functions and other types, and either come up
with assignments that satisfy these constraints, or, if possible, a proof that
the constraints aren't satisfiable. Over the last few years, SMT solvers using
DPLL(T)~\cite{dpllt:04} and other frameworks have come into their own and can
solve a wide variety of problems using efficient heuristics. Problems they can
attack range from simple puzzles like Sudoku and n-queens, to planning and
scheduling, program analysis~\cite{Gulwani:08}, whitebox fuzz
testing~\cite{Godefroid:08} and bounded model checking~\cite{Armando:09}.

Yet, SMT solvers are only used by a small number of experts. It isn't hard to
see why: the standard way for programs to interact with SMT solvers like
Z3~\cite{z3}, Yices~\cite{yices} and CVC3~\cite{cvc3} is via powerful but
relatively arcane C APIs that require the users to know the particular solver's
internals. For example, here is a C program that asks Z3 whether the simple
proposition $p \wedge \neg p$ is satisfiable.

\begin{verbatim}
Z3_config cfg = Z3_mk_config();
Z3_context ctx = Z3_mk_context(cfg);
Z3_del_config(cfg);
Z3_sort bool_sort = Z3_mk_bool_sort(ctx);

Z3_symbol symbol_p = Z3_mk_int_symbol(ctx, 0);

Z3_ast p = Z3_mk_const(ctx, symbol_p, bool_sort);
Z3_ast not_p = Z3_mk_not(ctx, p);

Z3_ast args[2] = {p, not_p};
Z3_ast conjecture = Z3_mk_and(ctx, 2, args);

Z3_assert_cnstr(ctx, conjecture);

Z3_lbool sat = Z3_check(ctx);

Z3_del_context(ctx);
return sat;
\end{verbatim}

Simultaneously, most SMT solvers also feature interaction via the standard input
language SMT-LIB~\cite{smtlib2:10}. SMT-LIB is \textit{significantly} easier to
use in isolation. The same program in SMT-LIB would look something like

\begin{verbatim}
; Declare a variable we don't know the value of yet
(declare-fun p () Bool)
; Try to find a value satisfying a contradiction
(assert (and p (not p)))
(check-sat)
; Prints "unsat", meaning "unsatisfiable"
\end{verbatim}

However, the SMT-LIB interfaces are generally hard to use directly from C
programs and often not as full-featured\footnote{Z3, for instance, supports
  plugging in external theories via the C API, but not via the textual SMT-LIB
  interface.}  or extensible. Importantly, it is difficult to write programs
that \textit{interact} dynamically with the solver in some way, for example by
adding assertions based on generated models, and therefore it is often difficult
to build new abstractions on top.

To overcome these difficulties, we decided to re-implement SMT-LIB in a way
that allowed for the same power as the C version while appearing naturally
integrated into a host language. Since SMT-LIB is {\em S-expression}-based, a
Lisp seemed like a natural choice for the host language. We chose
Racket~\cite{racket} for our implementation, \texttt{z3.rkt}, because of its
extensive facilities for implementing new languages~\cite{Tobin-Hochstadt:11},
not just for the interface to the solver, but also for the resulting program
analysis tools that the solver would make possible.

Using this system, the program above becomes almost as brief as the SMT-LIB
version.\footnote{The names are slightly different; these differences are
minor, systematic, and enumerated in Section~\ref{sec:porting-smt-lib}.}

\begin{verbatim}
(smt:with-context
 (smt:new-context-info)
 (smt:declare-fun p () Bool)
 (smt:assert (and/s p (not/s p)))
 (smt:check-sat))
\end{verbatim}

\section{Interactive SMT solving}

We now turn our attention to a problem that demonstrates how the interaction of
a language with an SMT solver is useful. A Sudoku puzzle asks the player to
complete a partially pre-filled 9$\times$9 grid with the numbers 1 through 9
such that no row, column, or 3$\times$3 box has two instances of a number. This
is a classic constraint satisfaction problem, and any constraint solver can
handle it with ease.

A Racket program using \texttt{z3.rkt} to solve Sudoku would look like the
following:

\begin{verbatim}
(define (solve-sudoku grid)
  (smt:with-context
   (smt:new-context-info)
   ;; Declare a scalar datatype (finite domain type) with 9 entries
   (smt:declare-datatypes () ((Sudoku S1 S2 S3 S4 S5 S6 S7 S8 S9)))
   ;; Represent the grid as an array from integers to the Sudoku type.
   (smt:declare-fun sudoku-grid () (Array Int Sudoku))
   ;; Assert the standard grid rules (row, column, box)
   (add-sudoku-grid-rules)
   ;; Add pre-filled entries
   (add-grid grid)
   (define sat (smt:check-sat))
   ;; 'sat means we found a solution, 'unsat means we didn't
   (if (eq? sat 'sat)
       ;; Retrieve the values from the model
       (for/list ([x (in-range 0 81)])
         (smt:eval (select/s sudoku-grid x)))
       #f)))
\end{verbatim}

Here we omit a couple of function definitions: \texttt{add-sudoku-grid-rules}
asserts the standard Sudoku grid rules, and \texttt{add-grid} reads the grid in
a particular format and creates assertions based on it. We note that the SMT-LIB
function \texttt{(select arr x)} retrieves the value at \texttt{x} from the
array \texttt{arr}, and that this can be used to add constraints on the array
(for instance, \texttt{(assert (= (select arr x) y))}).

However, simply finding a solution isn't enough for a good Sudoku solver: it
must also verify that there aren't any other solutions. The usual way to do that
for a constraint solver is by retrieving a generated model, adding assertions
such that this model cannot be generated, and then asking the solver whether the
system of assertions is still satisfiable. If they are, a second solution
exists and the puzzle is considered invalid.

This is where the interactivity offered by \texttt{z3.rkt} becomes useful, since
it lets the programmer add assertions on the fly. The last part of the program
then becomes

\begin{verbatim}
   ...
   (if (eq? sat 'sat)
       ;; Make sure no other solution exists
       (let ([result-grid
         (for/list ([x (in-range 0 81)])
           (smt:eval (select/s sudoku-grid x)))])
         ;; Assert that we want a brand new solution by asserting
         ;; (not <current solution>)
         (smt:assert
          (not/s (apply and/s
                        (for/list ([(x i) (in-indexed result-grid)])
                          (=/s (select/s sudoku-grid i) x)))))
         (if (eq? (smt:check-sat) 'sat)
             #f ; Multiple solutions
             result-grid))
       #f)))
\end{verbatim}

This part can even be abstracted out into a function that returns a lazily-
generated sequence of satisfying assignments for any given set of constraints.
