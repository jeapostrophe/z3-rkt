\section{Related Work}
Integration with SMT solvers is desirable for any programming
language. It enables programmers in that language to use the
power of SMT solvers to solve logical constraints arising in
a program, and use the solutions obtained in the rest of the
program. The alternative would be to hand code the solution
process in the program itself, which is time consuming and
inefficient for complex constraints. Thus, it is not
surprising that several such projects exist, most of them
available freely on the world wide web. These project differ
mainly in the host language, the format of the interface, and
the constructs they support.  We give details of some of
the implementations that use some functional language as the
host language and compare our approach with these
implementations.  

The integration of Scala with Z3~\cite{scalaz3} is one of the
most complete implementation. It provides support for adding
new theories and procedural abstraction that gives more
expressiveness to the system. It takes advantage of Scala's
type system to deal with type related errors at compile time.
The system is used in solving several challenging problems
within and outside the group that developed it. The main
disadvantage of this system is that the syntax is far from
SMT-LIB format, and is almost as verbose as the C APIs.

