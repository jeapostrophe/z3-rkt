\section{Related Work}
Integration with SMT solvers is desirable for any programming
language. It enables programmers in that language to use the
power of SMT solvers to solve logical constraints arising in
a program, and use the solutions obtained in the rest of the
program. The alternative would be to hand code the solution
process in the program itself, which is time consuming and
inefficient for complex constraints. Thus, it is not
surprising that several such projects exist, most of them
available freely on the world wide web. These project differ
mainly in the host language, the format of the interface, and
the constructs they support.  We give details of some of the
implementations and compare our approach with these
implementations.  Note that, as most languages support some
way of interaction with C functions, theoretically these
languages are already integrated with Z3 (or other SMT
solvers) through the C API calls. However, we do not consider
this as an integration because it does not simplify the task
of the programmer and, as noted in Section~\ref{sec:motiv},
they require the user to know the internals of Z3.


The integration of Scala with Z3~\cite{scalaz3} is one of the
most complete implementation. It provides support for adding
new theories and procedural abstraction that gives more
expressiveness to the system. It takes advantage of Scala's
type system to deal with type related errors at compile time.
The system is used in solving several challenging problems
within and outside the group that developed it. The main
disadvantage of this system is that the syntax is far from
SMT-LIB format, and is almost as verbose as using the C bindings.

Symbolic Bit Vectors (SBV)~\cite{sbv} is a package developed
in Haskell that can be used to prove properties about
bit-precise Haskell programs. Given a constraint in Haskell
program, SBV generate SMT-LIB formula that can be solved
using any SMT solver.  The major limitation of SBV is that it
works only for bit vector formulae. SBV has been hooked up
with Yices SMT solver~\cite{yices}.

Yices-Painless~\cite{yices-painless} is another project that
integrates Haskell with Yices SMT solver~\cite{yices}.  The
integration of Yices-Painless is done using Yices's C
API. Thus the integration is specific to Yices.  This project
does not yet support arrays, tuples, lists and user defined
data types, but supports bit vectors and uninterpreted
functions only. Further, the development of the tool seems to
be stalled for some time now (last change to the repository
was in January 2011).

Z3 Documentation page~\cite{z3} lists bindings for Z3
in other languages like OCAML and Python. These bindings,
however, are in almost one-to-one correspondence with the C
bindings. Thus, they suffer the same disadvantages as
mentioned for C bindings.
